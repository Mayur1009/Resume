\documentclass{my_cv}
\usepackage[skins]{tcolorbox}
\usepackage{hyperref}
\usepackage[none]{hyphenat}
\usepackage{lipsum}


\hypersetup{
	colorlinks=true,
	linkcolor=teal,
	urlcolor=teal
}

\urlstyle{same}
%\renewcommand{\baselinestretch}{1.2} 
\begin{document}
	
%	\noindent 
%	\begin{minipage}{0.5\textwidth}
%		\begin{flushleft}
%			{
%				\ralewayextra\bfseries\fontsize{40}{40}\selectfont Mayur Kishor Shende
%				
%			}
%		\end{flushleft}
%	\end{minipage}
%	\hspace{0.1\textwidth}
%	\begin{minipage}{0.4\textwidth}
%		\begin{flushright}
%			%    \vspace*{-0.1cm}
%			{\ralewayextra \href{}{\faEnvelope}} \textbf{mayur.k.shende@gmail.com}\\
%			{\ralewayextra \href{}{\faPhone}} \textbf{+91-8668955880}\\
%			{\ralewayextra \iconhref{orcid.org}{\faOrcid}} \textbf{\href{https://orcid.org/0000-0002-1738-2573}{0000-0002-1738-2573}}\\
%			{\ralewayextra \iconhref{https://www.linkedin.com/}{\faLinkedin}} \textbf{\url{linkedin.com/in/mayur1009}}\\
%			{\ralewayextra \iconhref{github.com}{\faGithub}} \textbf{\url{github.com/Mayur1009}}\\
%			{\ralewayextra \iconhref{gitlab.com}{\faGitlab}} \textbf{\url{gitlab.com/Mayur1009}}\\
%			{\ralewayextra \iconhref{scholar.google.com}{\aiGoogleScholarSquare} \textbf{\href{https://scholar.google.com/citations?user=FqYV6sAAAAAJ&hl=en}{FqYV6sAAAAAJ}}}\\
%			{\ralewayextra \iconhref{www.researchgate.net}{\aiResearchGateSquare} \textbf{\href{https://www.researchgate.net/profile/Mayur-Shende-2}{Mayur-Shende-2}}}
%
%		\end{flushright}
%	\end{minipage}
	

%\begin{multicols}{2}[
    \titletext{Mayur\\[12pt]\hspace*{1.4em}Shende}%
        {}%
        {}%
        {mayur.k.shende@gmail.com}%
        {+91-8668955880}%
        {\href{https://orcid.org/0000-0002-1738-2573}{0000-0002-1738-2573}}%
        {\url{linkedin.com/in/mayur1009}}%
        {\url{github.com/Mayur1009}}%
    	{\url{gitlab.com/Mayur1009} \\ %
    		\ralewayextra \iconhref{scholar.google.com}{\aiGoogleScholarSquare} \textbf{\href{https://scholar.google.com/citations?user=FqYV6sAAAAAJ&hl=en}{FqYV6sAAAAAJ}} \\
    		\ralewayextra \iconhref{www.researchgate.net}{\aiResearchGateSquare} \textbf{\href{https://www.researchgate.net/profile/Mayur-Shende-2}{Mayur-Shende-2}}
    	}%
%]
%\end{multicols}

%\vspace{1em}

%\section{\faFileText}{SUMMARY}
%My strong suits are Python development and machine (deep) learning in both industry and academia. Over the next few years, I intend to develop as a generalist software engineer. For that, I have learned a bit more of modern C++, frontend technologies, data engineering, DevOps, software architecture, and even product management as well.

\section{\faGraduationCap}{EDUCATION}

\begin{multicols}{2}

\school{Master of Technology (M. Tech)} %
{\faCalendar \: 2021-present}%
{Artificial Intelligence}%
{Defence Institute of Advanced Technology, Pune, India} %
{Department of Computer Science and Engineering}%
{7.75 / 10.00}%
{TBD}

\columnbreak

\school{Bachelor of Engineering (B.E.)} %
{\faCalendar \: 2017-2021}%
{}%
{Goverment College of Engineering, Nagpur, India} %
{Department of Computer Science and Engineering}%
{9.28 / 10.00} %
{Development of an R library for Automated Time-series Cleaning}

\end{multicols}


%\noindent\textbf{Tools:} PyTorch, OpenCV, NumPy, Pandas, Flask, PyTest, MQTT, Docker, Jenkins, Terraform

%\noindent\textbf{Other:} CI/CD, Version Control, Web, Data science and ML, Cloud computing

%Proactive about learning diverse things and happy to discuss those. Favorite physical activities would be cycling and hiking.

\section{\faList}{SKILLS}
	\keyword{Python} \keyword{R} \keyword{C++} \keyword{C} \keyword{Shiny} \keyword{Tensorflow} \keyword{LaTex} \keyword{HTML} \keyword{CSS} \keyword{Flutter} \keyword{GitHub} \keyword{Git} %\keyword{JavaScript} \keyword{NodeJS} \keyword{Autodesk Forge}


\section{\faPen}{EXPERIENCE}

%	\begin{multicols}{2}
	
		\work{Summer Internship - Visvesvaraya National Institute Of Technology, Nagpur}{\faCalendar \: May 2019 - August 2019}{Implemented PSF(Pattern Sequence Based Forecasting) Forecasting Algorithm, for univariate time-series forecasting, in Python (\url{https://pypi.org/project/PSF-Py/}). Also worked with tools for data visualization.}{Python, Matplotlib, Forecasting, ARIMA, Prophet}
		
		\work{Google Summer of Code (2021)}{\faCalendar \: May 2021 - August 2021}{Project link:  \url{https://summerofcode.withgoogle.com/archive/2021/projects/5676749848838144}\\The goal of this project was to develop a new R package, named \emph{cleanTS} (\url{https://cran.r-project.org/web/packages/cleanTS/index.html}). The package automates the proces of cleaning uniariate time-series data and provides new ways to visualize data in different resolutions.}{Data Cleaning, Univariate Time-Series, R, Shiny, Animated Visualizations}
		
%		\columnbreak
		
		\work{Winter Internship - Visvesvaraya National Institute Of Technology, Nagpur}{\faCalendar \: April 2020 - August 2020}{Worked on a implementation of Jaya, an optimization algorithm. An R package for the same was also published (\url{https://cran.rstudio.com/web/packages/Jaya/index.html}). Also, worked in the fields of data visualization, forecasting, and image processing and various tools for data manipulation in R.}{R, GA, Data Visualization, Optimization Algorithms}
		
		\work{Google Summer of Code (2022)}{\faCalendar \: May 2022 - November 2022}{Project link:  \url{https://summerofcode.withgoogle.com/programs/2022/projects/0WjqfO7k}\\The goal of the project is to develop a new R package, modifying the imputeTestbench package (data imputation) for Genomics applications with better computational capabilities.}{Data Imputation, Genomics, R, Shiny}

%	\end{multicol 	s}
%\work{Miscellaneous / Jul 2020 - Aug 2021}%
%    {}%
%    {Worked a few short stints at different organizations, including at my own non-profit startup. Focused on AI, DevOps, MLOps, and a bit of Agile Project Management.}
%    {Jenkins, Terraform, AWS, GCP, DVC + CML, GatsbyJs, ClickUp}
%
%\work{ML Engineer / Mar 2019 - May 2020}%
%    {Smart Cart Co - Delaware, US}%
%    {Researched novel approaches for large scale, yet fine grained visual classification. Enhanced code readability and performance by redesigning and implementing it in modules.}%
%    {Python, C++, GStreamer, DeepStream, CNN}
%
%\work{Python Developer / Mar 2018 - Feb 2019}%
%    {The LHC - Geneva, CH}%
%    {Worked on backend development in Python and on machine learning methods for textual data - from research to production.}%
%    {Python, Flask, PyTest, FastAI, PyTorch, scikit-learn, CNN, LSTM}
%
%\work{Summer Student / Jun 2017 - Sep 2017}%
%    {The LHC - Geneva, CH}%
%    {Configured and simulated runs of different detector-particle beam interactions. Added a more robust track reconstruction algorithm (General Broken Lines) to the Proteus framework}{}%
%


\section{\faStar}{PUBLICATIONS}

	\begin{itemize}
		\setlength\itemsep{1em}
		
		\item \textbf{Mayur Kishor Shende}, Andrés E. Feijóo-Lorenzo, Neeraj Dhanraj Bokde. \textbf{cleanTS: Automated (AutoML) tool to clean univariate time series at microscales}. \emph{Elsevier. \textbf{Neurocomputing}} Volume 500, Pages 155-176 (2022). \textbf{(IF 5.719)} (\url{https://doi.org/10.1016/j.neucom.2022.05.057}).\\
		\href{https://www.sciencedirect.com/science/article/pii/S0925231222006117/pdfft?isDTMRedir=true&download=true}{\faDownload} \keyword{Software} \keyword{Data Cleaning} \keyword{Time Series Analysis} \keyword{Machine Leaning} \keyword{AutoML}
		
		\item \textbf{Shende, M.K.}; Salih, S.Q.; Bokde, N.D.; Scholz, M.; Oudah, A.Y.; Yaseen, Z.M. \textbf{Natural Time Series Parameters Forecasting: Validation of the Pattern-Sequence-Based Forecasting (PSF) Algorithm; A New Python Package.} \textit{MDPI}. \textbf{\textit{Applied Sciences}} 2022, 12, 6194. \textbf{(IF 2.736)} \url{https://doi.org/10.3390/app12126194}. \\
		\href{https://www.mdpi.com/2076-3417/12/12/6194/pdf}{\faDownload} \keyword{Software} \keyword{Time Series Analysis} \keyword{Forecasting} \keyword{Machine leaning}
		
		\item M Sawant, \textbf{MK Shende}, AE Feijóo-Lorenzo, ND Bokde. \textbf{The State-of-the-Art Progress in Cloud Detection, Identification, and Tracking Approaches: A Systematic Review}. \emph{Multidisciplinary Digital Publishing Institute (MDPI). \textbf{Energies}} Volume 14 Issue 23. (2021) \textbf{(IF 3.004)} (\url{https://www.mdpi.com/1996-1073/14/23/8119}).\\
		\href{https://www.mdpi.com/1996-1073/14/23/8119/pdf}{\faDownload} \keyword{Review} \keyword{Cloud Detection} \keyword{Cloud Imaging} \keyword{Object Detection} \keyword{Image Processing}
		
		\item Agenis-Nevers M., Bokde N., Yaseen Z., and \textbf{Shende M.} (2020). \textbf{An empirical estimation for time and memory algorithm complexities: Newly developed R package}. \emph{Multimedia Tools and Application} \textbf{(IF 2.757)}. 80, 2997-3015 (\url{https://doi.org/10.1007/s11042-020-09471-8}).\\
		\href{https://link.springer.com/content/pdf/10.1007/s11042-020-09471-8.pdf}{\faDownload} \keyword{Software} \keyword{Complexity} \keyword{AutoML}
		%    \item My startup was awarded a top 5 position at the the 2020 Social Startup competition.
		%    \item Full scholarship and Gold Medal in Master studies.
		%    \item Selected as a volunteer at that amazing conference in 2018 held in Harrapa.
	\end{itemize}


\section{\faProjectDiagram}{PROJECTS}
	\begin{itemize}[noitemsep]
		\item Development of an R Package for automated time-series cleaning. (\url{github.com/Mayur1009/cleanTS}).
		\item Implementation of an optimization algorithm JAYA in R. (\url{https://cran.r-project.org/package=Jaya}).
		\item Created Python Package implementing PSF(Pattern Sequence Based Forecasting) algorithm. (\url{https://pypi.org/project/PSF-Py}).
		\item Contribution in data visualization in the R package \textit{ForecastTB} (\url{https://cran.r-project.org/package=ForecastTB}).
		\item Created a Flutter application for detection of crops from given image.
		\item Project to implement an algorithm that detects clouds from satellite images. This was part of problem statement given by ISRO in Smart India Hackathon, 2019.
		
		%    \item \textbf{Market research} and \textbf{literature review} of trends in digital mental healthcare.
		%    \item Design of a \textbf{pilot study} to study a semi-novel intervention in mental healthcare.
		%    \item Facilitation of agile implementation in teams - \textbf{SCRUM, Kanban}
		%    \item Fish detection and classification - \textbf{Probabilistic Modeling, CNN}.
		%    \item People counting in dense crowd images using sparse head detections - \textbf{CNN, SVM}.
		%    \item Vehicle detection, classification, and tracking - \textbf{CNN, openCV}.
		%    \item Simultaneous Localization and Mapping (SLAM) on a robotic wheelchair - \textbf{ROS}
	\end{itemize}

\section{\faCertificate}{CERTIFICATIONS}
	\noindent	
	\textbf{\aiCoursera\hspace{1mm} Machine Learning}
	\begin{itemize}
		\item By Stanford University on Coursera
		\item Logistic Regression
		\item Machine Learning Algorithms
	\end{itemize}
	
	\noindent
	\textbf{\aiCoursera\hspace{1mm} Deep Learning Specialization}
	\begin{itemize}
		\item By deeplearning.ai on Coursera
		\item Neural Networks and Deep Learning
		\item Improving Deep Neural Networks: Hyperparameter tuning, Regularization and Optimization
		\item Structuring Machine Learning Projects
		\item Convolutional Neural Networks(Ongoing)
		\item Sequence Models(Ongoing)
	\end{itemize}
	
	
	\noindent
	\textbf{\aiCoursera\hspace{1mm}Introduction to TensorFlow for Artificial Intelligence, Machine Learning, and Deep Learning}
	\begin{itemize}
		\item By deeplearning.ai on Coursera.
		\item Introduction to Neural networks.
		\item Implementing various networks using Tensorflow.
	\end{itemize}


\begin{multicols}{2}
		
	\section{\faBaseballBall}{ACTIVITIES AND INTERESTS}
		
		\noindent
		\textbf{Team Leader, Smart India Hackathon 2019}
		\begin{itemize}
			\item Qualified and selected from the college round.
			\item Problem Statement: Cloud Movement Prediction
		\end{itemize}
		
		\noindent
		\textbf{Event Organizer, SYNERGY, GCOEN, 2019}
		\begin{itemize}
			\item Inter College technical event “SYNERGY”
			\item Part of the organizing team for gaming events in Synergy, 2019.
		\end{itemize}
	
	\columnbreak
	
	\section{\faUserFriends}{REFEREE}
	
		\textbf{Dr. NEERAJ DHANRAJ BOKDE}
		\begin{itemize}
			\item Assistant Professor, \href{http://qgg.au.dk/en/}{Center for Quantitative Genetics and Genomics, Aarhus University, Aarhus, Denmark}
			\item Email: \url{neerajdhanraj@qgg.au.dk}
			\item Website: \url{https://www.neerajbokde.in/}
		\end{itemize}
	
\end{multicols}

\end{document}
